\newpage
\chapter{Conclusion}
\label{Conclusion}

Population growth and increasing urbanisation lead to more and more congestion problems in cities. This is not only inconvenient for the population, but also has large environmental and economic impact. The challenge for this project is to create a sustainable transport network that will reduce the congestion problems, but is also economically feasible. For this reason an Urban Air Mobility System is designed and implementation into the existing transport network is investigated. 

In this report, the commuting flows of Los Angeles have been investigated and visualised to find the most populated and congested routes. These routes will be the main target of the system design, as the most time profit can be obtained here, which results in a higher feasibility of the system succeeding. To find what kind of vehicle and network is optimal for servicing these routes, a tool was created to find values of the quantitative criteria. Multiple concepts have been designed, with different passenger capacities. Three ranges have been observed: 1-2, 4-6 and 20+ passengers. Intermediate trade-offs using only quantitative criteria have been performed to reduce the amount of concepts to one per passenger range. In the final trade-off that was performed, also the qualitative criteria were included and it was found that the Tilt Wing 4 (Bumblebee) is the winning concept. The concept beats the other concepts mostly on the battery mass, ticket price and safety criteria. 

A sensitivity analysis has been performed on the tool that evaluates the different concepts. Input parameters were changed and the effect on the outputs was analysed, which still found the four person vehicle as the top concept. Also the sensitivity of the weight factors was analysed, which confirmed that the robustness of the tool and showed Tilt Wing 4 (Bumblebee) as a deserved winner of the trade-off. 

Following this Mid Term Report, the winning concept will be worked out in more detail, regarding the vehicle, operations and infrastructure. The performance of the vehicle will be determined and detailed drawings of the lay-out will be created. The operations of the vertiports and landing pads will be characterised and there will be illustrations of the infrastructure. The network of the air mobility system will be described for the city of Los Angeles and the flexibility of the system is investigated by implementing the system in other cities. This Final Report will present the complete mobility system that should perform as a sustainable and affordable alternative of existing transport networks. 



